\documentclass{article}                    % article class
 
\begin{document}                           % Begin document text
 
\textbf{{ GOOGLE PLAY} }  
\section{ About}    
            % Print a section heading
Google Play (previously Android Market) is a digital distribution service operated and developed by Google. It serves as the official app store for the Android operating system, allowing users to browse and download applications developed with the Android software development kit (SDK) and published through Google. Google Play also serves as a digital media store, offering music, magazines, books, movies, and television programs. It previously offered Google hardware devices for purchase until the introduction of a separate online hardware retailer, Google Store, on March 11, 2015.

Applications are available through Google Play either free of charge or at a cost. They can be downloaded directly on an Android device through the Play Store mobile app or by deploying the application to a device from the Google Play website. Applications exploiting hardware capabilities of a device can be targeted to users of devices with specific hardware components, such as a motion sensor (for motion-dependent games) or a front-facing camera (for online video calling). The Google Play store had over 82 billion app downloads in 2016 and has reached over 3.5 million apps published in 2017.[4] It has been the subject of multiple issues concerning security, in which malicious software has been approved and uploaded to the store and downloaded by users, with varying degrees of severity.

Google Play was launched on March 6, 2012, bringing together the Android Market, Google Music, and the Google eBookstore under one brand, marking a shift in Google's digital distribution strategy. The services operating under the Google Play banner are: Google Play Books, Google Play Games, Google Play Movies and TV, Google Play Music, Google Play Newsstand, and Google Play Console. Following their re-branding, Google has gradually expanded the geographical support for each of the services.

\section{ Play Games}    
            % Print a section heading
Google Play Games is an online gaming service for Android that features real-time multiplayer gaming capabilities, cloud saves, social and public leaderboards, and achievements. The service was introduced at the Google I/O 2013 Developer Conference,[21] and the standalone mobile app was launched on July 24, 2013.[22].
\section{ Music}    
            % Print a section heading
Google Play Music is a music and podcast streaming service and online music locker. It features over 40 million songs,[23] and gives users free cloud storage of up to 50,000 songs.[24]

As of May 2017, Google Play Music is available in 64 countries].
\section{ books}    
            % Print a section heading
Google Play Books is an ebook digital distribution service. Google Play offers over five million ebooks available for purchase,[26] and users can also upload up to 1,000 of their own ebooks in the form of PDF or EPUB file formats.[27] As of January 2017, Google Play Books is available in 75 countries.[25] Google's Play Store now includes audiobooks. You can listen to your favorite books with a real person's storytelling, not by voice synthesis. Some books are narrated by their authors. With a large selection of books currently available in 45 countries].
\section{ history}    
            % Print a section heading
Google Play originated from three distinct products: Android Market, Google Music and Google eBookstore.

The Android Market was announced by Google on August 28, 2008,[34][35] and was made available to users on October 22.[36][37] In December 2010, content filtering was added to the Android Market, each app's details page started showing a promotional graphic at the top, and the maximum size of an app was raised from 25 megabytes to 50 megabytes.[38][39][40] The Google eBookstore was launched on December 6, 2010, debuting with three million ebooks, making it "the largest ebooks collection in the world".[41] In November 2011, Google announced Google Music, a section of the Play Store offering music purchases.[42][43] In March 2012, Google increased the maximum allowed size of an app by allowing developers to attach two expansion files to an app's basic download; each expansion file with a maximum size of 2 gigabytes, giving app developers a total of 4 gigabytes.[44][45] Also in March, the Android Market was re-branded as Google Play.[46][47][48]

In May 2016, it was announced that the Google Play Store, including all Android apps, would be coming to Chrome OS in September 2016.
\section{ User interface}    
            % Print a section heading
Apart from searching for content by name, apps can also be searched through keywords provided by the developer.[51] When searching for apps, users can press on suggested search filters, helping them to find apps matching the determined filters.[52] For the discoverability of apps, Play Store consists of lists featuring top apps in each category, including "Top Free", a list of the most popular free apps of all time; "Top Paid", a list of the most popular paid apps of all time; "Top Grossing", a list of apps generating the highest amounts of revenue; "Trending Apps", a list of apps with recent installation growth; "Top New Free", a list of the most popular new free apps; "Top New Paid", a list of the most popular new paid apps; "Featured", a list of new apps selected by the Google Play team; "Staff Picks", a frequently-updated list of apps selected by the Google Play team; "Editors' Choice", a list of apps considered the best of all time; and "Top Developer", a list of apps made by developers considered the best.[53] In March 2017, Google added a "Free App of the Week" section, offering one normally-paid app for free.[54][55] In July 2017, Google expanded its "Editors' Choice" section to feature curated lists of apps deemed to provide good Android experiences within overall themes, such as fitness, video calling and puzzle games.[56][57]

Google Play enables users to know the popularity of apps, by displaying the number of times the app has been downloaded. The download count is a color-coded badge, with special color designations for surpassing certain app download milestones, including grey for 100, 500, 1,000 and 5,000 downloads, blue for 10,000 and 50,000 downloads, green for 100,000 and 500,000 downloads, and red/orange for 1 million, 5 million, 10 million and 1 billion downloads.[58][59]

Users can submit reviews and ratings for apps and digital content distributed through Google Play, which are displayed publicly. Ratings are based on a 5-point scale. App developers can respond to reviews[60] using the Google Play Developer Console.[61]
\section{ Design}    
            % Print a section heading
Google has redesigned Google Play's interface on several occasions. In February 2011, Google introduced a website interface for then-named Android Market that provides access through a computer.[62] Applications purchased are downloaded and installed on an Android device remotely, with a "My Market Account" section letting users give their devices a nickname for easy recognition.[63] In May 2011, Google added new application lists to Android Market, including "Top Paid", "Top Free", "Editor's Choice", "Top Grossing", "Top Developers", and "Trending".[64][65] In July, Google introduced an interface with a focus on featured content, more search filters, and (in the US) book sales and movie rentals.[66] In May 2013, a redesign to the website interface matched the then-recently redesigned Android app.[67] In July 2014, the Play Store Android app added new headers to the Books/Movies sections, a new Additional Information screen offering a list featuring the latest available app version, installed size, and content rating, and simplified the app permissions prompt into overview categories.[68] A few days later, it got a redesign consistent with the then-new Material Design design language,[69][70] and the app was again updated in October 2015 to feature new animations, divide up the content into "Apps and Games" and "Entertainment" sections, as well as added support for languages read right-to-left.[71][72][73] In April 2016, Google announced a redesign of all the icons used for its suite of Play apps, adding a similar style and consistent look.[74][75] In May 2017, Google removed the shopping bag from the Google Play icon, with only the triangle and associated colors remaining.[76][77]
\section{ App monetization}    
            % Print a section heading
Google states in its Developer Policy Center that "Google Play supports a variety of monetization strategies to benefit developers and users, including paid distribution, in-app products, subscriptions, and ad-based models", and requires developers to comply with the policies in order to "ensure the best user experience". It requires that developers charging for apps and downloads through Google Play must use Google Play's payment system. In-app purchases unlocking additional app functionality must also use the Google Play payment system, except in cases where the purchase "is solely for physical products" or "is for digital content that may be consumed outside of the app itself (e.g. songs that can be played on other music players)."[78] Support for paid applications was introduced on February 13, 2009 for developers in the United States and the United Kingdom,[79] with support expanded to an additional 29 countries on September 30, 2010.[80] The in-app billing system was originally introduced in March 2011.[81] All developers on Google Play are required to feature a physical address on the app's page in Google Play, a requirement established in September 2014.[82]

In February 2017, Google announced that it would let developers set sales for their apps, with the original price struck out and a banner underneath informing users when the sale ends. Google also announced that it had made changes to its algorithms to promote games based on user engagement and not just downloads. Finally, it announced new editorial pages for what it considers "optimal gaming experiences on Android", further promoting and curating games.[12][13][14]
\section{ Play Store on Android}    
            % Print a section heading
Play Store is Google's official pre-installed app store on Android-certified devices. It provides access to content on the Google Play Store, including apps, books, magazines, music, movies, and television programs.[97]

Play Store filters the list of apps to those compatible with the user's device. Developers can target specific hardware components (such as compass), software components (such as widget), and Android versions (such as 7.0 Nougat).[98] Carriers can also ban certain apps from being installed on users' devices, for example tethering applications.[99]

There is no requirement that Android applications must be acquired using the Play Store. Users may download Android applications from a developer's website or through a third-party app store alternative.[100] Play Store applications are self-contained Android Package files (APK), similar to .exe files to install programs on Microsoft Windows computers.[101] On Android devices, an "Unknown sources" feature in Settings allows users to bypass the Play Store and install APKs from other sources.[102] Depending on developer preferences, some apps can be installed to a phone's external storage card.[103]

Android users have complained that the Google Play store access cannot be blocked and there is constant data exchange with the google cloud. Also valuable CPU ressources are used, slowing down the Android system.[104]
\section{ Installation history}    
            % Print a section heading
The Play Store app features a history of all installed apps. Users can remove apps from the list, with the changes also synchronizing to the Google Play website interface, where the option to remove apps from the history does not exist.[105]
\section{ Compatibility}    
            % Print a section heading
Google publishes the source code for Android through its "Android Open Source Project", allowing enthusiasts and developers to program and distribute their own modified versions of the operating system. However, not all these modified versions are compatible with apps developed for Google's official Android versions. The "Android Compatibility Program" serves to "define a baseline implementation of Android that is compatible with third-party apps written by developers". Only Android devices that comply with Google's compatibility requirements may install and access Google's Play Store application. As stated in a help page for the Android Open Source Project, "Devices that are "Android compatible" may participate in the Android ecosystem, including Android Market; devices that don't meet the compatibility requirements exist outside that ecosystem. In other words, the Android Compatibility Program is how we separate "Android compatible devices" from devices that merely run derivatives of the source code. We welcome all uses of the Android source code, but only Android compatible devices—as defined and tested by the Android Compatibility Program—may participate in the Android ecosystem."[106]

Some device manufacturers choose to use their own app store instead of—or in addition to—the Play Store. Examples include Amazon opting for Amazon Appstore instead of Google Play for its Kindle Fire tablet computers,[107] and Samsung adding Galaxy Apps for its line of Samsung Galaxy smartphones and tablets.[108]
\section{ References}    
            % Print a section heading
"Google Play Store". APKMirror. Android Police. March 11, 2018. Retrieved March 11, 2018.
"Google Play Store (Android TV)". APKMirror. Android Police. March 11, 2018. Retrieved March 11, 2018.
"Google Play Store (Android Wear)". APKMirror. Android Police. March 11, 2018. Retrieved March 11, 2018.
"Number of Google Play Store apps 2017 | Statistic". Statista. Retrieved 2018-01-03.
"Number of Android applications". AppBrain. February 9, 2017. Archived from the original on February 10, 2017. Retrieved February 24, 2017.
"Paid app availability". Google Play Help. Google. Retrieved February 24, 2017.
"Supported locations for developer and merchant registration". Developer Console Help. Google. Retrieved February 24, 2017.
"How to use the Google Play Developer Console". Developer Console Help. Google. Retrieved February 24, 2017.
"Supported locations for distribution to Google Play users". Developer Console Help. Google. Retrieved February 24, 2017.
"Set up prices and app distribution". Developer Console Help. Google. Retrieved February 24, 2017.
"Transaction fees". Developer Console Help. Google. Retrieved February 24, 2017.
Bankhead, Paul (February 27, 2017). "Welcome to Google Developer Day at Game Developer Conference 2017". Android Developers Blog. Google. Retrieved March 1, 2017.
Kastrenakes, Jacob (February 28, 2017). "Google now lets apps display sale prices in the Play Store". The Verge. Vox Media. Retrieved March 1, 2017.
Rossignol, Derrick (February 28, 2017). "Google now lets developers offer sales on Android apps". Engadget. AOL. Retrieved March 1, 2017.
"Set up alpha/beta tests". Developer Console Help. Google. Retrieved February 24, 2017.
"Release app updates with staged rollouts". Developer Console Help. Google. Retrieved February 24, 2017.
"Pre-order on Google Play". Google Play Help. Google. Retrieved March 1, 2017.
O'Brien, Terrence (May 2, 2012). "Google Play adds carrier billing for music, movies and books". Engadget. AOL. Retrieved February 23, 2017.
"Returns and refunds on Google Play". Google Play Help. Google. Retrieved February 24, 2017.
"Wear App Quality". Android Developers. Google. Retrieved February 24, 2017.
Webster, Andrew (May 15, 2013). "Google announces Play game services, Android's cross-platform answer to Game Center". The Verge. Vox Media. Retrieved January 18, 2017.
Ingraham, Nathan (July 24, 2013). "Google takes on Game Center with Google Play Games for Android". The Verge. Vox Media. Retrieved January 18, 2017.
Li, Abner (February 22, 2017). "Play Music 7.4 adds 'Recents' to navigation drawer, now has 40 million songs in library". 9to5Google. Retrieved February 24, 2017.
"How to use Google Play Music". Google Play Help. Google. Retrieved January 21, 2017.
"Country availability for apps and digital content". Google Play Help. Google. Retrieved May 20, 2017.
Etherington, Darrell (March 6, 2013). "Google Play Offers Over 5M eBooks And More Than 18M Songs, One Year After Its Rebranding". TechCrunch. AOL. Retrieved February 23, 2017.
m4tt (May 15, 2013). "Google Play Books enables user ebook uploads, Google Drive support". The Verge. Vox Media. Retrieved February 23, 2017.
"Google's Play Store now includes audiobooks". PlayStore.zone. PLAYSTORE. February 20, 2018. Retrieved March 14, 2018.
"Rent or buy movies and TV shows". Google Play Help. Google. Retrieved January 15, 2017.
"Subscriptions on Google Play". Google Play Help. Google. Retrieved January 18, 2017.
"Add news sources and topics to personalize Newsstand". Google Play Help. Google. Retrieved January 18, 2017.
Bowers, Andrew (March 11, 2015). "Meet the updated Chromebook Pixel and the new Google Store". Official Google Blog. Google. Retrieved February 24, 2017.
Amadeo, Ron (March 11, 2015). "Google launches the Google Store, a new place to buy hardware [Updated]". Ars Technica. Condé Nast. Retrieved February 24, 2017.
Shankland, Stephen (August 28, 2008). "Google announces Android Market for phone apps". CNET. CBS Interactive. Retrieved February 23, 2017.
Biggs, John (August 28, 2008). "Android to Get Its Own App Market". TechCrunch. AOL. Retrieved February 23, 2017.
Chu, Eric (October 22, 2008). "Android Market: Now available for users". Android Developers Blog. Google. Retrieved February 23, 2017.
Takahashi, Dean (October 22, 2008). "Google releases details on Android Market launch". VentureBeat. Retrieved February 23, 2017.
Burnette, Ed (December 11, 2010). "Big changes in store for Android Market". ZDNet. CBS Interactive. Retrieved February 23, 2017.
Hollister, Sean (December 11, 2010). "Android Market update streamlines content, nukes tabs, dismantles 24-hour return policy to appease devs". Engadget. AOL. Retrieved February 23, 2017.
Ziegler, Chris (November 24, 2010). "Android Market adding content ratings to all apps, past, present, and future". Engadget. AOL. Retrieved February 23, 2017.
Murray, Abraham (December 6, 2010). "Discover more than 3 million Google eBooks from your choice of booksellers and devices". Official Google Blog. Google. Retrieved February 23, 2017.
Miller, Ross (November 16, 2011). "Google Music store official: artist hubs, Google+ integration, and more". The Verge. Google. Retrieved February 23, 2017.
Rodriguez, Armando (November 16, 2011). "Get Started With Google's New Music Store". TechHive. International Data Group. Retrieved February 23, 2017.
Sadewo, Bams (March 6, 2012). "Google Increases App Size Limit on Android Market to 4GB". Android Authority. Retrieved February 23, 2017.
Albanesius, Chloe (March 6, 2012). "Google Ups Android App Size Limit to 4GB". PC Magazine. Ziff Davis. Retrieved February 23, 2017.
Rosenberg, Jamie (March 6, 2012). "Introducing Google Play: All your entertainment, anywhere you go". Official Google Blog. Google. Retrieved February 23, 2017.
Topolsky, Joshua (March 6, 2012). "Hello, Google Play: Google launches sweeping revamp of app, book, music, and video stores". The Verge. Vox Media. Retrieved February 23, 2017.
Velazco, Chris (March 6, 2012). "Goodbye Android Market, Hello Google Play". TechCrunch. AOL. Retrieved February 23, 2017.
Amadeo, Ron (May 19, 2016). "The Play Store comes to Chrome OS, but not the way we were expecting". Ars Technica. Condé Nast. Retrieved February 24, 2017.
Protalinski, Emil (May 19, 2016). "Google Play is coming to Chrome OS in September". VentureBeat. Retrieved February 24, 2017.
"Get discovered on Google Play search". Developer Console Help. Google. Retrieved February 24, 2017.
Whitwam, Ryan (June 6, 2017). "Play Store suggested search filters are rolling out to all". Android Police. Retrieved June 8, 2017.
"Types of featured app lists". Developer Console Help. Google. Retrieved February 24, 2017.
Ghoshal, Abhimanyu (March 27, 2017). "Google Play now offers a free Android app every week". The Next Web. Retrieved April 26, 2017.
Protalinski, Emil (March 26, 2017). "Google Play gets a Free App of the Week section". VentureBeat. Retrieved April 26, 2017.
Locklear, Mallory (July 19, 2017). "Google Play wants to help users find apps with curated lists". Engadget. AOL. Retrieved July 20, 2017.
Pelegrin, Williams (July 19, 2017). "Google Play's editorial pages put a human touch on finding high-quality apps". Android Authority. Retrieved July 20, 2017.
Stimac, Blake (August 12, 2014). "Google Play Store now showcases app download numbers with new badges". Greenbot. International Data Group. Retrieved February 24, 2017.
Grush, Andrew (August 12, 2014). "Play Store introduces colorful badges denoting number of app downloads". Android Authority. Retrieved February 24, 2017.
Blagdon, Jeff (May 14, 2013). "Google Play now lets all developers respond to user reviews". The Verge. Vox Media. Retrieved April 26, 2017.
"View and analyze your app's ratings and reviews". Developer Console Help. Google. Retrieved February 25, 2017.
Chu, Eric (February 2, 2011). "Introducing the Android Market website". Google Mobile Blog. Google. Retrieved February 23, 2017.
Savov, Vlad (February 2, 2011). "Android Market gets a web store with OTA installations, in-app purchases coming soon". Engadget. AOL. Retrieved February 23, 2017.
Cutler, Kim-Mai (May 11, 2011). "Android Market Now Highlights Top-Grossing and Trending Apps". Adweek. Beringer Capital. Retrieved February 23, 2017.
Camp, Jeffrey Van (May 12, 2011). "Google revamps Android Market, adds more lists". Digital Trends. Retrieved February 23, 2017.
Montoy-Wilson, Paul (July 12, 2011). "A new Android Market for phones, with books and movies". Google Mobile Blog. Google. Retrieved February 23, 2017.
Ingraham, Nathan (May 15, 2013). "Google redesigning Play apps and Play Store on the web". The Verge. Vox Media. Retrieved April 26, 2017.
Whitwam, Ryan (July 15, 2014).  Latest Google Play Store 4.8.22 With PayPal Support, Simplified App Permissions, Bigger Buttons, And More [APK Download]". Android Police. Retrieved February 23, 2017.
Ruddock, David (July 22, 2014). "Google Play Store Update 4.9.13 Adds Material Design App And Content Pages [APK Download]". Android Police. Retrieved February 24, 2017.
Walter, Derek (July 23, 2014). "Google Play Store gets a Material Design-inspired makeover". Greenbot. International Data Group. Retrieved February 24, 2017.
Vincent, James (October 16, 2015). "A first look at the Google Play store redesign". The Verge. Vox Media. Retrieved February 23, 2017.
Perez, Sarah (October 16, 2015). "Google Play Is Getting A Makeover". TechCrunch. AOL. Retrieved February 23, 2017.
Steele, Billy (October 16, 2015). "Google Play's pending redesign gets an early tease". Engadget. AOL. Retrieved February 23, 2017.
Whitwam, Ryan (April 4, 2016). "Google Announces New Google Play App Icons". Android Police. Retrieved February 23, 2017.
Statt, Nick (April 4, 2016). "Google Play app icons are getting the candy-colored flat design treatment". The Verge. Vox Media. Retrieved February 23, 2017.
Toombs, Cody (May 10, 2017). "The Play Store adopts new app and notification icons with v7.8.16 [APK Download]". Android Police. Retrieved June 2, 2017.
Carman, Ashley (May 10, 2017). "Google drops the shopping bag from the Play Store icon". The Verge. Vox Media. Retrieved June 2, 2017.
"Monetization and Ads". Developer Policy Center. Google. Retrieved February 25, 2017.
Chu, Eric (February 13, 2009). "Android Market update: support for priced applications". Android Developers Blog. Google. Retrieved February 23, 2017.
Chu, Eric (September 30, 2010). "More Countries, More sellers, More buyers". Android Developers Blog. Google. Retrieved February 23, 2017.
Chu, Eric (March 24, 2011). "In-App Billing on Android Market: Ready for Testing". Android Developers Blog. Google. Retrieved February 23, 2017.
Whitwam, Ryan (September 18, 2014). "Google Will Now Require All App Publishers With Paid Apps Or In-App Purchases To Have An Address On File And Displayed Publicly In Google Play [Update]". Android Police. Retrieved April 26, 2017.
"Accepted payment methods on Google Play". Google Play Help. Google. Retrieved May 29, 2017.
Ziegler, Chris (May 2, 2012). "Google Play now lets you charge movies, music, and books to your phone bill on some carriers". The Verge. Vox Media. Retrieved February 23, 2017.
Ruddock, David (May 15, 2014). "Play Store Now Accepts PayPal In 12 Countries, Carrier Billing, Gift Cards, And Developer Sales Countries Expanded, Too". Android Police. Retrieved February 23, 2017.
Seifert, Dan (May 15, 2014). "You can now use PayPal to buy apps from the Google Play Store". The Verge. Vox Media. Retrieved May 29, 2017.
Russakovskii, Artem (August 15, 2012). "Google Is Gearing Up To Finally Introduce Play Store Gift Cards And A Wishlist [APK Teardown]". Android Police. Retrieved April 27, 2017.
Nickinson, Phil (August 16, 2012). "Google Play gift cards are real - and here's what they look like". Android Central. Mobile Nations. Retrieved April 27, 2017.
Nickinson, Phil (August 21, 2012). "Google Play Gift Cards are official, rolling out over the next few weeks". Android Central. Mobile Nations. Retrieved April 27, 2017.
Lee, Michael (August 22, 2012). "Google rolls out Google Play gift cards". ZDNet. CBS Interactive. Retrieved February 23, 2017.
"Where to buy Google Play gift cards". Google Play Help. Google. Retrieved April 27, 2017.

\end{document}


