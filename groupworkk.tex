\documentclass{article}                    % article class
 
\begin{document}                           % Begin document text
 
\textbf{{ RESEARCH PROPOSAL} }  

\section{  Title: }    
Evaluation of the effects of climate change on rice productivity and adaptation strategies in Eastern part of Uganda(kibimba)
\section{ introduction}
            % Print a section heading
Recent studies show that Uganda is one of the most vulnerable continents to climate variability and change and low adaptive capacity. It had been noted that the high dependence of the economies and rural people of Kibimba Uganda upon rain-fed agriculture, prevalence of poverty, food insecurity and limited development of institutional and infrastructural capacities in this region make coping with natural climate variability a perennial challenge . The largest population which resides in rural areas with livelihoods largely dependent on agriculture are highly vulnerable to climate change effects. With nearly seventy percent of Uganda's population dependent on agriculture and the sector contributing nearly forty percent of the country's GDP, Uganda remains vulnerable to climate variability and long term climate change. Some adaptation to current climate variability has been taking place; however, this may be insufficient for future changes in climate . 
Although recently, climate change issues are receiving a lot of empirical and documentary attention, especially as they affect rural areas of developing countries, there have been relatively insufficient discussions engaging with the science of climate change impact on agriculture and with the specification of smallholder and subsistence systems . It had been predicted that many farmers in Kibimba Uganda are likely to experience net revenue losses as a result of climate change, particularly as a result of increased variability and extreme events . This prediction needs to be investigated empirically so as to ascertain the level of such losses if any, especially as it concerns rice production in Uganda. Rice is a very important food crop in Uganda but domestic rice production has not been able to match the increased demand and hence Ugandan Government spends heavily on importation of rice and this constitutes a huge drain on the country's foreign reserve . 
Though there is evidence of increase in food crop production generally in Uganda, the reality is that the nation is not self sufficient in production of food crop except coffee . The question remains therefore as to whether the production level will ever meet the demand level given the rate of population growth in the country. Also, the proportion of change in production due to impact of climate change will remain an important research focus as well as measures needed to improve the resilience of the farmers to enable them adapt to climate change. Against this foregoing backdrop this study was designed to investigate the magnitudes of climate change effects on rice productivity and adaptation measures of small scale rice farmers in Easter part of Uganda(Kibimba rice scheme).
\end{document}

